\section{Εισαγωγή}
Σε αυτό το εργαστήριο καλούμαστε να πετύχουμε τη σύγκλιση της θέσης του κινητήρα στη θέση $\theta_{ref} = \SI{5}{\volt}$
με την ύπαρξη (σταθερών) διαταραχών μέσω του μαγνητικού φρένου.

Όπως και στη θεωρία, προσθέτουμε τον ολοκληρωτή:
\[\dot{z} = y - r = C x - r = x_1 - \theta_{ref}\]
και θα χρησιμοποιήσουμε ελεγκτή της μορφής:
\[u = -k_1 x_1 - k_2 x_2 - k_i z\]
\subsection{Ανάλυση Συστήματος}
\newcommand{\kone}{\frac{-k_m}{T_m s + 1}}
\newcommand{\ktwo}{k_{\mu}}
\newcommand{\kthree}{\frac{k_0}{s}}
\begin{figure}[htbp]
    \centering
    {
  \tikzstyle{block} = [draw, rectangle, minimum height=3em]
  \tikzstyle{sum} = [draw, fill=blue!20, circle, node distance=1cm]
  \tikzstyle{input} = [coordinate]
  \tikzstyle{output} = [coordinate]
  \tikzstyle{pinstyle} = [pin edge={to-, thin, black}]
  \tikzset{line/.style={draw, -latex}}

  % Width calculated with inkscape.
  \begin{tikzpicture}[auto, scale=\linewidth/14.78cm, transform shape, node distance=2cm,>=latex']
    % We start by placing the blocks
    \node [input, name=input] {};
    \node [sum, right of=input] (sum_r) {$+$};
    \node [block, right of=sum_r, node distance=1cm] (integral) {$\int$};
    \node [block, right of=integral] (k_i) {$k_i$};
    \node [sum, right of=k_i, node distance=1.5cm] (sum) {$+$};
    \node [block, right of=sum, node distance=3.5cm] (controller1) {$\kone$};
    \node [block, right of=controller1, node distance=2.5cm] (controller2) {$\ktwo$};
    \node [block, right of=controller2, node distance=1cm] (controller3) {$\kthree$};
    \draw [line] (controller1) -- node[name=Omega] {$\Omega$} (controller2);
    \node [output, right of=controller3] (output) {};
    \node [block, below of=controller1] (k_tacho) {$-k_T$};
    \node [block, left of=k_tacho, node distance=2.2cm] (k2) {$k_2$};
    \node [block, below of=k2] (k1) {$k_1$};
    \node [sum] (sumbelow) at (k2 -| sum) {$+$};

    % Once the nodes are placed, connecting them is easy.
    \draw [line] (input) node [left] {$r$} -- (sum_r.west) node [above, xshift=-0.1cm] {$-$};
    \draw [line] (sum_r) -- (integral);
    \draw [line] (integral) -- node {$z\left( t \right)$} (k_i);
    \draw [line] (k_i) -- (sum.west) node [above, xshift=-0.1cm] {$-$};
    \draw [line] (sum) -- node {$e$} (controller1);
    \draw [line] (controller2) -- node {} (controller3);
    \draw [line] (controller3) -- node [name=theta] {$x_1 = \theta$}(output);
    \draw [line] (Omega) |- (k_tacho);
    \draw [line] (k_tacho) -- node[label={[below]$x_2$}, label={$v_{tacho}$}] {} (k2);
    \draw [line] (theta) |- (k1);
    % Path from theta output to first sum with r.
    \path [line] (theta.south) --
    ([yshift=-5.5cm]theta.south) --
    ([yshift=-5.5cm]sum_r.center) --
    (sum_r.south) node [left, yshift=-0.1cm] {$+$};
    % Second sum node.
    \draw [line] (k1) -| (sumbelow.south) node [left, yshift=-0.1cm] {$+$};
    \draw [line] (k2) -- (sumbelow.east) node [above, xshift=0.1cm] {$+$};
    \draw [line] (sumbelow.north) -- (sum.south) node [left, yshift=-0.1cm] {$-$};
  \end{tikzpicture}
}

%%% Local Variables:
%%% mode: latex
%%% TeX-master: "../lab3"
%%% End:

    \caption{Το σύστημα με δυναμική ανάδραση καταστάσεων}\label{fig:system-feedback-dynamic}
\end{figure}
Η ανάλυση του συστήματος θα βασιστεί στο δομικό διάγραμμα του σχήματος~\ref{fig:system-feedback-dynamic}.
Όπως και στην ανάλυση του προηγούμενου εργαστηρίου προκύπτει η σχέση:
\begin{equation}
  \label{eq:x_2-1}
  e k_m k_T = T_m s x_2 + x_2
\end{equation}
όπου $\dot{x}_2 = s x_2$ και
\begin{equation}%TODO: d(t)??
  \label{eq:e}
  e = - k_i z - k_1 x_1 - k_2 x_2
\end{equation}
Από~\ref{eq:x_2-1} και~\ref{eq:e}:
\begin{align*}
  k_m k_T (-k_i z - k_1 x_1 -k_2 x_2) &= T_m \dot{x}_2 + x_2 \implies\\
  \dot{x}_2 &= -\frac{k_1 k_m k_T}{T_m} x_1 -\frac{k_2 k_m k_T + 1}{T_m} x_2 - \frac{k_i k_m k_T}{T_m} z
\end{align*}
και, όπως και στο προηγούμενο εργαστήριο:
\begin{equation}
  \label{eq:x_1}
  \dot{x}_1 = -\frac{k_{\mu} k_0}{k_T} x_2
\end{equation}
Οι πίνακες του συστήματος προκύπτουν:
\begin{align}
  A & = \begin{bmatrix}
    0 & -\frac{k_{\mu} k_0}{k_T} & 0\\
    -\frac{k_1 k_m k_T}{T_m} & \frac{1 + k_2 k_m k_T}{T_m} & -\frac{k_i k_m k_T}{T_m}\\
    1 & 0 & 0
  \end{bmatrix} \\
  B & = \begin{bmatrix}
    0\\
    0\\
    -1
  \end{bmatrix}
\end{align}
και το χαρακτηριστικό πολυώνυμο:
\begin{equation}
  \label{eq:sI-A}
  s^3 + \frac{k_2 k_T k_m + 1}{T_m} s^2 -\frac{k_0 k_1 k_m k_{\mu}}{T_m} s - \frac{k_0 k_i k_m k_{\mu}}{T_m}
\end{equation}
Για επιθυμητό πολυώνυμο $s^3 + a_1 s^2 + s_2 s + a_3$ προκύπτουν οι σχέσεις:
\begin{align}
  \label{eq:ks}
  k_1 &= -\frac{a_2 T_m}{k_0 k_m k_{\mu}}\\
  k_2 &= \frac{a_1 T_m - 1}{k_T k_m}\\
  k_i &= -\frac{a_3 T_m}{k_o k_m k_{\mu}}
\end{align}
και για τους πόλους ισχύει:
\begin{align}
  a_1 &= p_1 + p_2 + p_3\\
  a_2 &= p_1 p_2 + p_2 p_3 + p_1 p_3\\
  a_3 &= p_1 p_2 p_3
\end{align}
%%% Local Variables:
%%% mode: latex
%%% TeX-master: "../lab3"
%%% End:
